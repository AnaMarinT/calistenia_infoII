\documentclass[12pt,letterpaper]{article}
\usepackage[utf8]{inputenc}
\usepackage[spanish]{babel}
\usepackage{graphicx}
\graphicspath{ {images/} }

\begin{document}

\begin{titlepage}
    \begin{center}
        \vspace*{1cm}
            
        \Huge
        \textbf{Calistenia informática II}
            
        \vspace{0.5cm}
        \LARGE
        
    
        \vspace{1.5cm}
            
        \textbf{Ana María Marín Toro}
            
        \vfill
            
        \vspace{0.8cm}
            
        \Large
        Despartamento de Ingeniería Electrónica y Telecomunicaciones\\
        Universidad de Antioquia\\
        Medellín\\
        Marzo de 2021
            
    \end{center}

\end{titlepage}

\tableofcontents
\newpage
    \section {Introducción}
    El siguiente ejercicio consiste en redactar las instrucciones que debe seguir una persona con el fin de realizar una estructura piramidal utilizando dos tarjetas y un a hoja de papel, añadiendo la dificultad de realizarlo utilizando una sola mano. Los propósitos de esta actividad son,  por un lado enfatizar en lo problemático que resulta la implementación de un algoritmo que tiene ambigüedades, además de servir como practica para el uso de repositorios (GitHub) y de la herramienta Latex (Overleaf).
    
    \section{Condiciones iniciales}
    Lea cuidadosamente y asegúrese de cumplir con todos los requerimientos:\\\\
    a) Hoja de papel lisa.\\\\
    b) Dos tarjetas del mismo tamaño y peso aproximado.\\\\
    c) Se debe realizar el ejercicio sobre una superficie lisa y plana.\\\\
    d) Parta de la posición inicial: la hoja se encuentra extendida en la mesa, las tarjetas se encuentran a un lado tal como se ve en la siguiente imagen.\\\\

    \begin{figure}[h]
        \centering
        \includegraphics[width=10cm]{pinicial.png}
    \end{figure}
    
    \section{Instrucciones}
    Lea cada numeral y realice lo que se le pide antes de pasar al siguiente.\\\\
    1. Use la mano con la que sea más ágil, la otra ubíquela detrás de la espalda para evitar usarla por accidente.\\\\
    2. Utilice los dedos de la mano seleccionada, tome una de las tarjetas y ubíquela sobre la otra, de manera que estas queden alineadas.\\\\
    3. Utilice los dedos y la mano para levantar ambas tarjetas, asegúrese de mantenerlas alineadas.\\\\
    4. Pare las tarjetas sobre el centro geométrico de la hoja de papel, asegúrese de usar como base el lado más pequeño, además cerciórese de ver de frente los lados mas delgados de las tarjetas.\\\\
    5. Ubique los dedos pulgar y medio en las esquinas superiores de las tarjetas.\\\\
    6. Manteniendo la tarjeta que se encuentra mas alejada de la palma de la mano en la posición del numeral anterior, utilice los dedos pulgar y medio para mantener las tarjetas unidas por la parte superior, mientras que con los dedos restantes mueve lentamente la parte inferior de la otra tarjeta acercándola a la palma de su mano.\\\\
    7. Haga el movimiento del paso anterior hasta que considere que las tarjetas se encuentran equilibradas y forman una pirámide.\\\\
    8. Retire la mano de forma cuidadosa.\\\\

\end{document}


